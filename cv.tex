%%%%%%%%%%%%%%%%%%%%%%%%%%%%%%%%%%%%%%%%%
% Medium Length Professional CV
% LaTeX Template
% Version 2.0 (8/5/13)
%
% This template has been downloaded from:
% http://www.LaTeXTemplates.com
%
% Original author:
% Trey Hunner (http://www.treyhunner.com/)
%
% Important note:
% This template requires the resume.cls file to be in the same directory as the
% .tex file. The resume.cls file provides the resume style used for structuring the
% document.
%
%%%%%%%%%%%%%%%%%%%%%%%%%%%%%%%%%%%%%%%%%

%----------------------------------------------------------------------------------------
%	PACKAGES AND OTHER DOCUMENT CONFIGURATIONS
%----------------------------------------------------------------------------------------

\documentclass{resume} % Use the custom resume.cls style

\usepackage[left=0.4 in,top=0.4in,right=0.4 in,bottom=0.4in]{geometry} % Document margins
\newcommand{\tab}[1]{\hspace{.2667\textwidth}\rlap{#1}} 
\newcommand{\itab}[1]{\hspace{0em}\rlap{#1}}
\name{Беляев Станислав Валерьевич}
\address{Студент, Программист}
\address{Россия, Санкт-Петербург}
\address{+7 (981) 851-20-69 \\ \href{mailto:stasbelyaev96@gmail.com}{Email} \\ \href{https://github.com/StasBel}{GitHub} \\ \href{https://www.linkedin.com/in/stasbel/}{LinkedIn} \\ \href{https://hatscripts.com/addskype/?stasbelyaev96@gmail.com}{Skype}}

\begin{document}
\hfill {\scriptsize многое кликабельно}\\
%-------------------------------------------------------------------------
%	ОБРАЗОВАНИЕ
%-------------------------------------------------------------------------
\begin{rSection}{Образование}

{\bf Бакалавриат (3-й курс)} \hfill {Сентябрь 2014 - По наст. время}
\\ 
\href{http://spbau.ru/}{Санкт-Петербургский Академический Университет}
\\
\href{http://mit.spbau.ru/}{Кафедра математических и информационных технологий}
\\
\href{http://mit.spbau.ru/machine_learning}{Машинное обучение и анализ данных}

{\bf Летняя школа по байесовским методам в глубинном обучении} \hfill {Август 2017}
\\
\href{https://www.hse.ru/}{Высшая Школа Экономики}
\\
\href{https://cs.hse.ru/bayesgroup/}{Исследовательская группа байесовских методов}
\\
\url{http://deepbayes.ru/} 


\end{rSection}
%------------------------------------------------------------------------
%	НАВЫКИ
%------------------------------------------------------------------------

\begin{rSection}{Навыки}

\begin{tabular}{ @{} >{\bfseries}l @{\hspace{6ex}} l }
ЯП & Python, C++, Java, Kotlin, Ocaml, Haskell, R, Matlab \\ 
Технологии & git, bash, sql, gdb, docker, vagrant, jupyter \\
Интересы & Reinforcement Learning, Deep Learning, Neural Networks, Data Science \\  
\end{tabular}

\end{rSection}

%-------------------------------------------------------------------------%	ОПЫТ РАБОТЫ
%-------------------------------------------------------------------------

\begin{rSection}{Опыт работы}

\begin{rSubsection}{\href{http://stepik.org/}{Stepik (JetBrains Internship)}}{Июль 2017 - По наст. время}
{\href{https://jetbrains.ru/students/internship/themes/again/}{Разработка системы для кластеризациия code submissions, умные подсказки для обучения}}{} 
\item[] \url{https://github.com/StepicOrg/submissions-clustering}
\end{rSubsection}

\end{rSection} 


%-------------------------------------------------------------------------
%	ИЗБРАННЫЕ ПРОЕКТЫ И НИР'Ы
%-------------------------------------------------------------------------

\begin{rSection}{Избранные проекты и нир'ы} \itemsep -3pt  

\begin{rSubsection}{Prophet}{Сентябрь 2016 - Декабрь 2016}{\href{https://drive.google.com/file/d/0B6udzTbX1EFPTFZMSS1RVUVvXzg/view?usp=sharing}{Автоматическая генерация патчей методами ML}}{}
\item[] \url{https://github.com/StasBel/prophet-test}
\end{rSubsection}

\begin{rSubsection}{Compiler's course}{Декабрь 2016}{Реализация сборщика мусора на Ocaml}{}
\item[] \url{https://github.com/StasBel/compilers-homework}
\end{rSubsection}

\begin{rSubsection}{Program Synthesis}{Январь 2017 - Май 2017}{\href{https://docs.google.com/presentation/d/113EFcW8L7p8ickhfMoht8ivOomq2fRPizLtWSH9cSX4/pub?start=false&loop=false&delayms=3000}{Encoder-Decoder модель для генерации патчей}}{}
\item[] \url{https://github.com/StasBel/program-synthesis}
\end{rSubsection}

\end{rSection}

%-------------------------------------------------------------------------
%	ПУБЛИЧНЫЕ ВЫСТУПЛЕНИЯ
%-------------------------------------------------------------------------

\begin{rSection}{Публичные выступления} \itemsep -3pt  

{\href{https://docs.google.com/presentation/d/1ieE0JZWKbCQH_qpH81M1pSbUSARBP7DKLdGVwk563pM/pub?start=false&loop=false&delayms=3000}{Bayesian Sketch Learning for Program Synthesis}} \hfill May 2017 \\ 
{\href{https://docs.google.com/presentation/d/19FyHBksjffXY6nF1B1xRvkkW-lSp0QIPzU_o-h5RD00/pub?start=false&loop=false&delayms=3000}{Stepik Task}} \hfill June 2017 \\
\end{rSection} 

%-------------------------------------------------------------------------
%	ОНЛАЙН КУРСЫ
%-------------------------------------------------------------------------

\begin{rSection}{Онлайн курсы} \itemsep -3pt  

\begin{rSubsection}{Coursera}{}{}{} 
\item[] \fbox{\href{https://www.coursera.org/learn/vvedenie-mashinnoe-obuchenie}{Введение в машинное обучение}}
\end{rSubsection}

\begin{rSubsection}{Stepik}{}{}{} 
\item[] \fbox{\href{https://stepik.org/course/73}{Введение в Linux}}, \fbox{\href{https://stepik.org/course/95}{Математический анализ}}, \fbox{\href{https://stepik.org/course/253}{Архитектура ЭВМ, Элементы операционных систем}} \\
\fbox{\href{https://stepik.org/course/401/}{Нейронные сети}}
\end{rSubsection}

\end{rSection} 

\end{document}
